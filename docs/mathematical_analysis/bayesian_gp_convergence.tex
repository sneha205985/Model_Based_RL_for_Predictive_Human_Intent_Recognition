\section{Bayesian Gaussian Process Convergence Analysis}

This section provides a comprehensive mathematical analysis of Bayesian Gaussian Process convergence properties specifically tailored for human behavior modeling in the context of predictive intent recognition. We establish both theoretical convergence guarantees and practical performance bounds for real-time deployment.

\subsection{Bayesian Framework for Human Behavior Modeling}

\subsubsection{Problem Formulation}

Consider the human behavior modeling problem where we seek to learn a mapping $f^*: \mathcal{X} \rightarrow \mathcal{Y}$ from observable state features $\mathcal{X} \subseteq \R^d$ to human intent predictions $\mathcal{Y} \subseteq \R^p$. The Bayesian GP framework treats $f^*$ as a random function with prior distribution $f^* \sim \GP(m_0, k_0)$.

\begin{definition}[Human Behavior GP Model]
\label{def:human_behavior_gp}
The human behavior Gaussian Process is defined by:
\begin{align}
f(x) &\sim \GP(m_0(x), k_0(x, x')) \\
y_i &= f(x_i) + \epsilon_i, \quad \epsilon_i \sim \Normal(0, \sigma_n^2)
\end{align}
where $m_0(x) = \E[f(x)]$ is the prior mean function and $k_0(x, x')$ is the covariance kernel encoding our beliefs about human behavior smoothness and structure.
\end{definition}

\begin{assumption}[Human Behavior Regularity]
\label{ass:human_behavior_regularity}
The true human behavior function $f^*$ belongs to the RKHS $\mathcal{H}_{k_0}$ with bounded norm:
$$\norm{f^*}_{\mathcal{H}_{k_0}} \leq B_h$$
for some human behavior complexity bound $B_h > 0$.
\end{assumption}

\subsubsection{Posterior Inference and Bayesian Updates}

\begin{theorem}[GP Posterior for Human Behavior]
\label{thm:gp_posterior_human}
Given observations $\mathcal{D}_n = \{(x_i, y_i)\}_{i=1}^n$, the GP posterior for human behavior is:
\begin{align}
f | \mathcal{D}_n &\sim \GP(m_n(x), k_n(x, x')) \\
m_n(x) &= m_0(x) + k_n(x)^T (K_n + \sigma_n^2 I)^{-1} (\mathbf{y} - \mathbf{m}_0) \\
k_n(x, x') &= k_0(x, x') - k_n(x)^T (K_n + \sigma_n^2 I)^{-1} k_n(x')
\end{align}
where $k_n(x) = [k_0(x, x_1), \ldots, k_0(x, x_n)]^T$ and $K_n$ is the $n \times n$ covariance matrix.
\end{theorem>

\subsection{Convergence Analysis for Human Behavior Learning}

\subsubsection{Almost-Sure Convergence}

\begin{theorem}[Bayesian GP Almost-Sure Convergence]
\label{thm:bayesian_gp_as_convergence}
Under Assumption \ref{ass:human_behavior_regularity}, if the training inputs $\{x_i\}_{i=1}^\infty$ are dense in a compact set $\mathcal{K} \subset \mathcal{X}$, then:
$$\lim_{n \to \infty} m_n(x) = f^*(x) \quad \text{almost surely}$$
for all $x \in \mathcal{K}$, and the convergence is uniform on $\mathcal{K}$.
\end{theorem>

\begin{proof}
\textbf{Step 1: Bayesian consistency framework}
By Bayesian consistency theory, if the true function $f^*$ is in the support of the GP prior, then the posterior mean converges almost surely to $f^*$.

\textbf{Step 2: RKHS embedding}
Since $f^* \in \mathcal{H}_{k_0}$ by assumption, the GP prior assigns positive probability to neighborhoods of $f^*$.

\textbf{Step 3: Dense sampling argument}
The density of $\{x_i\}$ in $\mathcal{K}$ ensures that the posterior receives sufficient information to distinguish $f^*$ from other functions in $\mathcal{H}_{k_0}$.

\textbf{Step 4: Uniform convergence}
Uniform convergence follows from the compactness of $\mathcal{K}$ and the continuity properties of the kernel $k_0$.
\end{proof>

\subsubsection{Finite-Sample Convergence Rates}

\begin{theorem}[Bayesian GP Convergence Rate]
\label{thm:bayesian_gp_rate}
For the RBF kernel $k_0(x, x') = \sigma_f^2 \exp(-\frac{\norm{x - x'}^2}{2\ell^2})$ and i.i.d. training inputs, the posterior mean satisfies:
$$\E_{x \sim \mu}\E_{\mathcal{D}_n}[|m_n(x) - f^*(x)|^2] = \mathcal{O}(n^{-\frac{2s}{2s + d}})$$
where $s$ is the smoothness parameter of $f^*$ and $d$ is the input dimension.
\end{theorem>

\begin{proof}
This follows from the minimax theory for GP regression in RKHS. The rate $n^{-\frac{2s}{2s + d}}$ is optimal for functions with smoothness $s$ in dimension $d$.
\end{proof>

\subsubsection{Posterior Concentration}

\begin{theorem}[Posterior Concentration for Human Behavior GP]
\label{thm:posterior_concentration}
The GP posterior concentrates around the true function $f^*$ at the rate:
$$\Pi(f : \norm{f - f^*}_{\mathcal{H}_{k_0}} > M_n \mid \mathcal{D}_n) \to 0 \quad \text{a.s.}$$
where $M_n = \mathcal{O}(\sqrt{\frac{\log n}{n}})$ is the concentration rate.
\end{theorem>

\begin{proof}
\textbf{Step 1: Apply Bayesian concentration theory}
For GP priors satisfying certain regularity conditions, posterior concentration occurs at rates determined by the metric entropy of the RKHS ball.

\textbf{Step 2: Entropy calculation}
For RBF kernels, the metric entropy of RKHS balls scales as $\log N(\epsilon, \mathcal{H}_{k_0}) \sim \epsilon^{-d/s}$.

\textbf{Step 3: Concentration rate}
Combining entropy bounds with posterior concentration theory yields the stated rate.
\end{proof>

\subsection{Uncertainty Quantification and Calibration}

\subsubsection{Bayesian Credible Intervals}

\begin{theorem}[Bayesian Credible Intervals for Human Behavior]
\label{thm:bayesian_credible_intervals}
For the GP posterior, the $(1-\alpha)$ Bayesian credible interval at point $x$ is:
$$CI_{1-\alpha}(x) = [m_n(x) - z_{\alpha/2}\sqrt{k_n(x, x)}, m_n(x) + z_{\alpha/2}\sqrt{k_n(x, x)}]$$

Under correct specification, these intervals are exactly calibrated:
$$\Pr[f^*(x) \in CI_{1-\alpha}(x) \mid \mathcal{D}_n] = 1 - \alpha$$
\end{theorem}

\subsubsection{Practical Calibration Under Model Misspecification}

\begin{theorem}[Robust Calibration Analysis]
\label{thm:robust_calibration}
When the true human behavior model deviates from the GP assumption by $\delta_{\text{misspec}}$, the calibration error is bounded:
$$\left|\Pr[f^*(x) \in CI_{1-\alpha}(x) \mid \mathcal{D}_n] - (1-\alpha)\right| \leq \frac{C \delta_{\text{misspec}}}{\sqrt{k_n(x, x)}}$$
for some constant $C$ depending on the kernel parameters.
\end{theorem>

\subsection{Active Learning and Information Gain}

\subsubsection{Information-Theoretic Analysis}

\begin{definition}[Information Gain for Human Behavior Learning]
\label{def:info_gain_human}
The information gain from observing at location $x$ is:
$$I(x; \mathcal{D}_n) = \frac{1}{2} \log\left(1 + \frac{k_n(x, x)}{\sigma_n^2}\right)$$
\end{definition>

\begin{theorem}[Cumulative Information Gain Bound]
\label{thm:cumulative_info_gain}
For the RBF kernel with length scale $\ell$ in dimension $d$, the cumulative information gain after $n$ observations is bounded:
$$\gamma_n := \sum_{i=1}^n I(x_i; \mathcal{D}_{i-1}) \leq \frac{1}{2} \log\det(I + \sigma_n^{-2} K_n) = \mathcal{O}((\log n)^{d+1})$$
\end{theorem>

\subsubsection{Optimal Experimental Design}

\begin{theorem}[Bayesian Optimal Design for Human Behavior]
\label{thm:optimal_design}
The optimal next query point for learning human behavior is:
$$x_{n+1}^* = \argmax_{x \in \mathcal{X}} \sigma_n(x) = \argmax_{x \in \mathcal{X}} \sqrt{k_n(x, x)}$$

This greedy strategy achieves near-optimal information gain:
$$I(x_{n+1}^*; \mathcal{D}_n) \geq \frac{1}{2} \max_{x \in \mathcal{X}} I(x; \mathcal{D}_n) - \log(2)$$
\end{theorem>

\subsection{Multi-Output Human Behavior Modeling}

\subsubsection{Multi-Task GP Framework}

For predicting multiple aspects of human behavior simultaneously:

\begin{definition}[Multi-Output Human Behavior GP]
\label{def:multi_output_gp}
The multi-output GP for $p$ human behavior tasks is:
$$\mathbf{f}(x) = [f_1(x), \ldots, f_p(x)]^T \sim \GP(\mathbf{m}_0(x), K_{\text{multi}}(x, x'))$$
where $K_{\text{multi}}$ is a $p \times p$ matrix-valued kernel encoding task correlations.
\end{definition}

\begin{theorem}[Multi-Task Convergence]
\label{thm:multi_task_convergence}
For the separable kernel $K_{\text{multi}}(x, x') = K_{\text{task}} \otimes k_0(x, x')$, each task converges at the same rate as the single-output case:
$$\E[|m_{n,j}(x) - f_j^*(x)|^2] = \mathcal{O}(n^{-\frac{2s}{2s + d}})$$
for each task $j = 1, \ldots, p$.
\end{theorem>

\subsection{Computational Aspects and Scalability}

\subsubsection{Sparse GP Approximations}

\begin{theorem}[Sparse GP Convergence]
\label{thm:sparse_gp_convergence}
Using $m$ inducing points for sparse GP approximation, the approximation error is:
$$\E[\norm{m_n^{\text{sparse}} - m_n}_\infty^2] \leq \frac{C \sigma_f^2}{m^{2s/d}}$$
where $s$ is the kernel smoothness and $d$ is the dimension.
\end{theorem>

\subsubsection{Online Bayesian Updates}

\begin{theorem}[Online GP Update Complexity]
\label{thm:online_gp_complexity}
Online Bayesian updates for the GP model have computational complexity:
$$\mathcal{O}(n^2) \text{ for each update, } \mathcal{O}(n^3) \text{ for batch processing}$$

With sparse approximations using $m$ inducing points: $\mathcal{O}(m^2)$ per update.
\end{theorem>

\subsection{Robustness and Adaptation}

\subsubsection{Adaptive Kernel Learning}

\begin{theorem}[Kernel Parameter Adaptation]
\label{thm:kernel_adaptation}
Using empirical Bayes (Type-II ML) to adapt kernel parameters $\boldsymbol{\theta} = \{\sigma_f, \ell, \sigma_n\}$:
$$\hat{\boldsymbol{\theta}}_n = \argmax_{\boldsymbol{\theta}} p(\mathbf{y} \mid \mathbf{X}, \boldsymbol{\theta})$$

The adapted parameters converge to optimal values:
$$\hat{\boldsymbol{\theta}}_n \to \boldsymbol{\theta}^* \quad \text{as } n \to \infty$$
under regularity conditions.
\end{theorem>

\subsubsection{Non-Stationary Human Behavior}

\begin{theorem}[Time-Varying Human Behavior]
\label{thm:time_varying_behavior}
For non-stationary human behavior with gradual change rate $\rho$, the GP with forgetting factor $\lambda$ achieves tracking error:
$$\E[|m_n(x, t) - f^*(x, t)|^2] \leq \frac{C \rho}{1 - \lambda} + \mathcal{O}(n^{-1})$$
\end{theorem>

\subsection{Integration with Model Predictive Control}

\subsubsection{GP-MPC Uncertainty Propagation}

\begin{theorem}[Uncertainty Propagation in GP-MPC]
\label{thm:gp_mpc_uncertainty}
When the GP posterior mean $m_n(x)$ is used in MPC, the prediction uncertainty propagates as:
$$\sigma_{\text{MPC}}^2(k) \leq \sigma_n^2(x_0) \prod_{i=0}^{k-1} (1 + L_f^2 T_s^2)$$
where $L_f$ is the Lipschitz constant of the dynamics and $T_s$ is the sampling time.
\end{theorem>

\subsubsection{Safe Learning with Confidence Bounds}

\begin{theorem}[GP Confidence Bounds for Safe MPC]
\label{thm:gp_confidence_bounds}
With confidence parameter $\beta_t = 2\log(\pi^2 t^2/(6\delta))$, the GP confidence bounds:
$$|f^*(x) - m_n(x)| \leq \beta_t \sigma_n(x)$$
hold with probability $1 - \delta$. Using these bounds for constraint tightening in MPC ensures safety.
\end{theorem>

\subsection{Performance Metrics and Validation}

\subsubsection{Predictive Performance Measures}

\begin{definition}[Human Behavior Prediction Metrics]
\label{def:prediction_metrics}
Key metrics for evaluating GP human behavior models:
\begin{align}
\text{RMSE} &= \sqrt{\frac{1}{n_{\text{test}}} \sum_{i=1}^{n_{\text{test}}} (f^*(x_i) - m_n(x_i))^2} \\
\text{NLPD} &= -\frac{1}{n_{\text{test}}} \sum_{i=1}^{n_{\text{test}}} \log p(y_i^* \mid x_i, \mathcal{D}_n) \\
\text{Calibration Error} &= \left|\frac{1}{n_{\text{test}}} \sum_{i=1}^{n_{\text{test}}} \mathbb{I}[y_i^* \in CI_\alpha(x_i)] - (1-\alpha)\right|
\end{align}
\end{definition>

\begin{theorem}[Asymptotic Optimality of GP Predictions]
\label{thm:asymptotic_optimality}
As $n \to \infty$, the GP achieves optimal performance in all metrics:
\begin{align}
\text{RMSE} &\to 0 \\
\text{NLPD} &\to \min_{\text{all models}} \text{NLPD} \\
\text{Calibration Error} &\to 0
\end{align}
\end{theorem>

\subsection{Real-Time Implementation Guarantees}

\subsubsection{Computational Bounds}

\begin{theorem}[Real-Time GP Computation]
\label{thm:realtime_gp}
For real-time human behavior prediction with time budget $T_{\text{budget}}$:
\begin{enumerate}
    \item Exact GP inference: $\mathcal{O}(n^3)$ preprocessing, $\mathcal{O}(n^2)$ per prediction
    \item Sparse GP approximation: $\mathcal{O}(m^3)$ preprocessing, $\mathcal{O}(m^2)$ per prediction
    \item Local GP approximation: $\mathcal{O}(k^3)$ per prediction with $k \ll n$ local points
\end{enumerate>

For $T_{\text{budget}} = 10$ms and $n = 1000$, sparse approximation with $m = 100$ is required.
\end{theorem>

\subsubsection{Memory Efficiency}

\begin{theorem}[GP Memory Requirements]
\label{thm:gp_memory}
Memory requirements for GP human behavior modeling:
\begin{align}
\text{Exact GP:} &\quad \mathcal{O}(n^2 d) \text{ storage} \\
\text{Sparse GP:} &\quad \mathcal{O}(nm + m^2) \text{ storage} \\
\text{Online GP:} &\quad \mathcal{O}(w^2 d) \text{ with sliding window size } w
\end{align}
\end{theorem>

This comprehensive Bayesian GP convergence analysis provides the theoretical foundation for reliable human behavior learning in the Model-Based RL system, ensuring both statistical guarantees and practical implementability for real-time human-robot interaction scenarios.