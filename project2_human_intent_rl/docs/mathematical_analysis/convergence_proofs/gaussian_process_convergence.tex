\subsection{Gaussian Process Convergence Analysis}

This section establishes the convergence properties of Gaussian Processes for learning human behavior patterns in the context of intent recognition.

\subsubsection{Problem Setup and Assumptions}

\begin{assumption}[Regularity of True Function]
\label{ass:regularity}
The true human behavior function $f^* : \mathcal{S} \rightarrow \R$ belongs to the reproducing kernel Hilbert space (RKHS) $\mathcal{H}_k$ associated with kernel $k$, i.e., $f^* \in \mathcal{H}_k$ with $\norm{f^*}_{\mathcal{H}_k} \leq B$ for some $B > 0$.
\end{assumption}

\begin{assumption}[Bounded Observations]
\label{ass:bounded_obs}
The observations are bounded: $|y_i| \leq M$ for all $i = 1, \ldots, n$ and some $M > 0$.
\end{assumption}

\begin{assumption}[Sub-Gaussian Noise]
\label{ass:subgaussian}
The observation noise $\epsilon_i = y_i - f^*(x_i)$ is sub-Gaussian with parameter $\sigma^2$, i.e., $\E[\exp(t\epsilon_i)] \leq \exp(\sigma^2 t^2/2)$ for all $t \in \R$.
\end{assumption}

\subsubsection{Posterior Convergence}

\begin{theorem}[GP Posterior Convergence]
\label{thm:gp_convergence}
Under Assumptions \ref{ass:regularity}, \ref{ass:bounded_obs}, and \ref{ass:subgaussian}, let $\{(x_i, y_i)\}_{i=1}^n$ be training data where $x_i \in \mathcal{S}$ are drawn i.i.d. from a distribution $\mu$ with density bounded below by $\rho > 0$ on a compact set $\mathcal{K} \subset \mathcal{S}$.

Then, for the GP posterior mean $\hat{f}_n(x)$ at any point $x \in \mathcal{K}$:
$$\lim_{n \to \infty} \hat{f}_n(x) = f^*(x) \quad \text{almost surely}$$

Furthermore, the convergence is uniform on $\mathcal{K}$:
$$\lim_{n \to \infty} \sup_{x \in \mathcal{K}} |\hat{f}_n(x) - f^*(x)| = 0 \quad \text{almost surely}$$
\end{theorem}

\begin{proof}
The proof follows from the consistency theory of kernel methods in RKHS. 

\textbf{Step 1: Express GP posterior mean}
The GP posterior mean can be written as:
$$\hat{f}_n(x) = k_n(x)^T (K_n + \sigma_n^2 I)^{-1} \mathbf{y}_n$$
where $k_n(x) = [k(x, x_1), \ldots, k(x, x_n)]^T$, $K_n$ is the $n \times n$ covariance matrix with $(K_n)_{ij} = k(x_i, x_j)$, and $\mathbf{y}_n = [y_1, \ldots, y_n]^T$.

\textbf{Step 2: Decompose the error}
For any $x \in \mathcal{K}$:
\begin{align}
|\hat{f}_n(x) - f^*(x)| &= |k_n(x)^T (K_n + \sigma_n^2 I)^{-1} \mathbf{y}_n - f^*(x)| \\
&\leq |k_n(x)^T (K_n + \sigma_n^2 I)^{-1} \mathbf{f}_n^* - f^*(x)| \\
&\quad + |k_n(x)^T (K_n + \sigma_n^2 I)^{-1} \boldsymbol{\epsilon}_n|
\end{align}
where $\mathbf{f}_n^* = [f^*(x_1), \ldots, f^*(x_n)]^T$ and $\boldsymbol{\epsilon}_n = \mathbf{y}_n - \mathbf{f}_n^*$.

\textbf{Step 3: Bound the approximation error}
Since $f^* \in \mathcal{H}_k$, by the representer theorem and properties of RKHS:
$$|k_n(x)^T (K_n + \sigma_n^2 I)^{-1} \mathbf{f}_n^* - f^*(x)| \leq C \cdot \sigma_n^2 \cdot \norm{f^*}_{\mathcal{H}_k}$$
for some constant $C > 0$.

\textbf{Step 4: Bound the noise term}
Using the sub-Gaussian property of noise and concentration inequalities:
$$|k_n(x)^T (K_n + \sigma_n^2 I)^{-1} \boldsymbol{\epsilon}_n| \leq \sigma \sqrt{\frac{2\log(2/\delta)}{n}} \cdot \norm{k_n(x)^T (K_n + \sigma_n^2 I)^{-1}}$$
with probability at least $1-\delta$.

\textbf{Step 5: Apply law of large numbers}
Since $x_i$ are i.i.d. from $\mu$ with bounded density, as $n \to \infty$:
- $\sigma_n^2 \to 0$ by design (regularization parameter)
- The noise term vanishes by the strong law of large numbers
- The approximation quality improves due to dense sampling

Therefore, $\hat{f}_n(x) \to f^*(x)$ almost surely for each $x \in \mathcal{K}$.

\textbf{Step 6: Uniform convergence}
Uniform convergence follows from the equicontinuity of the family $\{\hat{f}_n\}$ (due to kernel smoothness) and the compactness of $\mathcal{K}$.
\end{proof}

\subsubsection{Convergence Rates}

\begin{theorem}[GP Convergence Rate]
\label{thm:gp_rate}
Under the conditions of Theorem \ref{thm:gp_convergence}, if the kernel $k$ is $\alpha$-smooth (i.e., the RKHS has smoothness index $\alpha$), then:
$$\E[\sup_{x \in \mathcal{K}} |\hat{f}_n(x) - f^*(x)|^2] = \mathcal{O}(n^{-\frac{2\alpha}{2\alpha + d}})$$
where $d$ is the effective dimension of the input space.
\end{theorem}

\begin{proof}
This follows from minimax theory in nonparametric regression. The proof uses:
1. Empirical process theory to bound the supremum
2. Metric entropy arguments for the function class
3. Concentration inequalities for kernel methods

The rate $n^{-\frac{2\alpha}{2\alpha + d}}$ is minimax optimal for this problem class.
\end{proof}

\subsubsection{Adaptive Learning Rates}

For practical implementation, we consider adaptive learning where the regularization parameter $\sigma_n^2$ is chosen data-dependently.

\begin{theorem}[Adaptive GP Convergence]
\label{thm:adaptive_gp}
Consider the regularization parameter choice:
$$\sigma_n^2 = \left(\frac{\log n}{n}\right)^{\frac{2\alpha}{2\alpha + d}}$$

Then, under the conditions of Theorem \ref{thm:gp_convergence}:
$$\sup_{x \in \mathcal{K}} |\hat{f}_n(x) - f^*(x)| = \mathcal{O}_p\left(\left(\frac{\log n}{n}\right)^{\frac{\alpha}{2\alpha + d}}\right)$$
\end{theorem}

\subsubsection{Posterior Uncertainty Quantification}

\begin{theorem}[GP Uncertainty Bounds]
\label{thm:gp_uncertainty}
The GP posterior variance provides valid uncertainty quantification:
$$\Pr\left[|f^*(x) - \hat{f}_n(x)| \leq \Phi^{-1}(1-\alpha/2) \sqrt{\hat{\sigma}_n^2(x)}\right] \geq 1 - \alpha$$
for any $\alpha \in (0, 1)$, where:
$$\hat{\sigma}_n^2(x) = k(x,x) - k_n(x)^T (K_n + \sigma_n^2 I)^{-1} k_n(x)$$
\end{theorem}

\begin{proof}
This follows from the Gaussian nature of the GP posterior and the calibration properties of Bayesian inference under correct model specification.
\end{proof}

\subsubsection{Implications for Human Behavior Learning}

\begin{corollary}[Human Behavior Model Consistency]
In the context of human intent recognition, if human behavior follows a smooth function $f^*: \mathcal{S} \rightarrow \mathcal{H}$ in the RKHS of the chosen kernel, then the GP model will converge to the true human behavior pattern with the rates established above.
\end{corollary}

This theoretical foundation ensures that the GP component of our system will reliably learn human behavior patterns given sufficient data, providing both point estimates and uncertainty quantification that improve over time.