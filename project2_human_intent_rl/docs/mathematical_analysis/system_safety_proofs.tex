\section{Integrated System Safety Verification}

This section provides comprehensive mathematical proofs for the safety properties of the complete integrated Model-Based RL system for human-robot interaction. We establish formal safety guarantees that account for the interaction between GP learning, MPC control, and RL adaptation components.

\subsection{Integrated System Architecture}

\subsubsection{System Composition}

The integrated system consists of three main components operating in closed-loop:

\begin{definition}[Integrated HRI System]
\label{def:integrated_hri_system}
The complete system is defined as:
$$\Sigma_{\text{HRI}} = \langle \text{GP}_{\text{human}}, \text{MPC}_{\text{control}}, \text{RL}_{\text{adapt}}, \mathcal{I} \rangle$$
where:
\begin{itemize}
    \item $\text{GP}_{\text{human}}$: Gaussian Process for human behavior prediction
    \item $\text{MPC}_{\text{control}}$: Model Predictive Controller with safety constraints
    \item $\text{RL}_{\text{adapt}}$: Reinforcement learning agent for policy adaptation
    \item $\mathcal{I}$: Information flow and integration protocols
\end{itemize}
\end{definition>

\subsubsection{System Dynamics with Integration}

The integrated system dynamics are:
\begin{align}
\label{eq:integrated_dynamics}
\dot{x}(t) &= f(x(t), u_{\text{MPC}}(t), h_{\text{human}}(t), w(t)) \\
u_{\text{MPC}}(t) &= \kappa_N(x(t), \hat{h}_{\text{GP}}(t), \pi_{\text{RL}}(t)) \\
\hat{h}_{\text{GP}}(t) &= m_{\text{GP}}(x(t)) \pm \beta_t \sigma_{\text{GP}}(x(t)) \\
\pi_{\text{RL}}(t) &= \text{Policy}(\text{Experience}_{1:t}, \text{Rewards}_{1:t})
\end{align}

\subsection{Safety Constraint Framework}

\subsubsection{Hierarchical Safety Constraints}

\begin{definition}[Multi-Level Safety Constraints]
\label{def:multilevel_safety}
The safety constraints are organized hierarchically:

\textbf{Level 1 - Critical Safety (Hard Constraints):}
$$\mathcal{C}_1 = \{x \in \mathcal{X} : g_i^{\text{crit}}(x) \leq 0, \quad i = 1, \ldots, n_1\}$$

\textbf{Level 2 - Operational Safety (Soft Constraints):}
$$\mathcal{C}_2 = \{x \in \mathcal{X} : g_j^{\text{oper}}(x) \leq \epsilon_j, \quad j = 1, \ldots, n_2\}$$

\textbf{Level 3 - Performance Optimization:}
$$\mathcal{C}_3 = \{x \in \mathcal{X} : g_k^{\text{perf}}(x) \leq \delta_k(t), \quad k = 1, \ldots, n_3\}$$
\end{definition>

\begin{assumption}[Safety Constraint Properties]
\label{ass:safety_constraints}
The safety constraints satisfy:
\begin{enumerate}
    \item \textbf{Continuity}: All constraint functions $g_i$ are continuously differentiable
    \item \textbf{Boundedness}: Gradients are bounded: $\norm{\nabla g_i(x)} \leq L_g$ 
    \item \textbf{Feasibility}: The safe set $\mathcal{S} = \cap_{i=1}^{n_1} \{x : g_i^{\text{crit}}(x) \leq 0\}$ has non-empty interior
    \item \textbf{Observability}: Safety-relevant states are observable with bounded noise
\end{enumerate}
\end{assumption}

\subsection{Probabilistic Safety Analysis}

\subsubsection{Joint Uncertainty Propagation}

\begin{theorem}[Integrated System Probabilistic Safety]
\label{thm:integrated_prob_safety}
For the integrated system with GP uncertainty $\sigma_{\text{GP}}(x)$, MPC robustness margin $\epsilon_{\text{MPC}}$, and RL exploration noise $\sigma_{\text{RL}}$, the probability of safety constraint satisfaction is:

$$\Pr[\forall t \in [0,T], x(t) \in \mathcal{S}] \geq 1 - \delta_{\text{total}}$$

where:
$$\delta_{\text{total}} \leq \delta_{\text{GP}} + \delta_{\text{MPC}} + \delta_{\text{RL}} + \delta_{\text{interaction}}$$

and $\delta_{\text{interaction}}$ captures coupling effects between components.
\end{theorem>

\begin{proof}
\textbf{Step 1: Component-wise uncertainty bounds}
Each component provides probabilistic guarantees:
- GP: $\Pr[|h_{\text{true}} - \hat{h}_{\text{GP}}| \leq \beta_t \sigma_{\text{GP}}] \geq 1 - \delta_{\text{GP}}$
- MPC: $\Pr[x \in \mathcal{S} | \text{constraints satisfied}] \geq 1 - \delta_{\text{MPC}}$  
- RL: $\Pr[\text{exploration bounded}] \geq 1 - \delta_{\text{RL}}$

\textbf{Step 2: Constraint tightening}
Tighten safety constraints to account for uncertainties:
$$\tilde{g}_i(x) = g_i(x) + \gamma_i(\beta_t \sigma_{\text{GP}}(x) + \epsilon_{\text{MPC}} + \sigma_{\text{RL}}) \leq 0$$

\textbf{Step 3: Union bound application}
Apply union bound over failure modes:
$$\Pr[\text{safety violation}] \leq \sum_{\text{components}} \Pr[\text{component failure}] + \Pr[\text{interaction failure}]$$

\textbf{Step 4: Bound interaction terms}
The interaction failure probability is bounded by analyzing coupling between components through sensitivity analysis.
\end{proof>

\subsubsection{Time-Varying Safety Bounds}

\begin{theorem}[Time-Varying Safety Guarantees]
\label{thm:time_varying_safety}
As the GP learns and RL converges, the safety bounds improve over time:
$$\Pr[x(t) \in \mathcal{S}] \geq 1 - \delta_0 e^{-\lambda t} - \delta_{\infty}$$
where $\delta_0 > 0$ captures initial uncertainty, $\lambda > 0$ is the learning rate, and $\delta_{\infty}$ is the residual uncertainty.
\end{theorem>

\begin{proof}
\textbf{Step 1: GP uncertainty decay}
From GP convergence analysis: $\sigma_{\text{GP}}(x, t) \leq \sigma_0 e^{-\alpha t} + \sigma_{\text{noise}}$

\textbf{Step 2: RL exploration decay}  
From RL theory: $\sigma_{\text{RL}}(t) \leq C t^{-\beta}$ for appropriate exploration schedule

\textbf{Step 3: MPC adaptation}
As model uncertainty decreases, MPC constraint tightening can be relaxed

\textbf{Step 4: Combine decay rates}
The overall safety probability improves with the slowest component decay rate.
\end{proof>

\subsection{Reachability Analysis for Integrated System}

\subsubsection{Forward Reachability with Learning}

\begin{theorem}[Forward Reachability Under Learning]
\label{thm:forward_reachability_learning}
For the integrated system starting from initial set $\mathcal{X}_0$, the forward reachable set over $[0,T]$ satisfies:
$$\mathcal{R}([0,T], \mathcal{X}_0) \subseteq \mathcal{R}_{\text{nominal}}([0,T], \mathcal{X}_0) \oplus \mathcal{E}_{\text{uncertainty}}(T)$$
where $\mathcal{E}_{\text{uncertainty}}(T)$ is the uncertainty set and $\oplus$ denotes Minkowski sum.
\end{theorem>

\begin{proof}
\textbf{Step 1: Nominal trajectory computation}
Compute nominal reachable set assuming perfect predictions:
$$\mathcal{R}_{\text{nominal}} = \{x(T) : x(0) \in \mathcal{X}_0, \dot{x} = f(x, u^*, h^*)\}$$

\textbf{Step 2: Uncertainty set construction}
Construct uncertainty set from component uncertainties:
$$\mathcal{E}_{\text{uncertainty}}(T) = \{e : \norm{e} \leq C_1 \sup_{t \in [0,T]} \sigma_{\text{GP}}(t) + C_2 \epsilon_{\text{MPC}} + C_3 \sigma_{\text{RL}}\}$$

\textbf{Step 3: Sensitivity analysis}
Use system Lipschitz constants to bound trajectory deviations from nominal due to uncertainties.
\end{proof>

\subsubsection{Backward Reachability for Safety Verification}

\begin{theorem}[Backward Reachability Safety Verification]
\label{thm:backward_reachability_safety}
The set of safe initial conditions is:
$$\mathcal{X}_{\text{safe}} = \{x_0 : \mathcal{R}([0,T], \{x_0\}) \cap \mathcal{U}_{\text{unsafe}} = \emptyset\}$$

This set can be computed by solving the Hamilton-Jacobi equation:
$$\frac{\partial V}{\partial t} + \min_{u} \max_{h,w} \left\{\nabla V \cdot f(x,u,h,w)\right\} = 0$$
with terminal condition $V(x,T) = \min_i g_i^{\text{crit}}(x)$.
\end{theorem>

\subsection{Barrier Functions for Integrated Safety}

\subsubsection{Adaptive Control Barrier Functions}

\begin{definition}[Adaptive Control Barrier Function]
\label{def:adaptive_cbf}
For the integrated system, define the adaptive control barrier function:
$$B(x,t) = \min_i \left\{g_i^{\text{crit}}(x) + \gamma_i(t) \sigma_{\text{total}}(x,t)\right\}$$
where $\gamma_i(t)$ are time-varying safety margins and $\sigma_{\text{total}}$ captures total system uncertainty.
\end{definition>

\begin{theorem}[Adaptive Barrier Function Safety]
\label{thm:adaptive_barrier_safety}
If the control law satisfies:
$$u(t) \in \argmax_{u \in \mathcal{U}} \left\{\dot{B}(x,t) + \alpha(B(x,t))\right\}$$
for class $\mathcal{K}$ function $\alpha$, then the system remains safe: $B(x(t),t) \geq 0$ for all $t \geq 0$.
\end{theorem>

\begin{proof}
\textbf{Step 1: Barrier condition verification}
The time derivative of the barrier function is:
$$\dot{B}(x,t) = \nabla_x B \cdot f(x,u,h,w) + \frac{\partial B}{\partial t}$$

\textbf{Step 2: Uncertainty handling}
The adaptive margin $\gamma_i(t) \sigma_{\text{total}}(x,t)$ ensures that even with uncertainties:
$$\dot{B}(x,t) \geq -\alpha(B(x,t))$$

\textbf{Step 3: Forward invariance}
This ensures forward invariance of the safe set $\{x : B(x,t) \geq 0\}$.
\end{proof>

\subsubsection{Multi-Agent Barrier Functions}

\begin{theorem}[Multi-Agent Safety with Barrier Functions]
\label{thm:multiagent_barrier}
For $N$ agents with individual barrier functions $B_i(x_i, x_{-i}, t)$, if each agent satisfies:
$$\dot{B}_i + \alpha_i(B_i) \geq 0$$
then inter-agent safety is guaranteed with probability at least:
$$1 - \sum_{i<j} \Pr[B_i(x_i, x_j, t) < 0 \text{ or } B_j(x_j, x_i, t) < 0]$$
\end{theorem>

\subsection{Formal Verification Methods}

\subsubsection{Model Checking for HRI Systems}

\begin{theorem}[Temporal Logic Safety Verification]
\label{thm:temporal_logic_safety}
Express safety properties in Computation Tree Logic (CTL):
$$\phi_{\text{safety}} = \text{AG}(\text{safe\_state}) \land \text{EF}(\text{goal\_state})$$

The integrated system satisfies $\phi_{\text{safety}}$ if:
\begin{enumerate}
    \item All reachable states are in $\mathcal{S}$ (safety)  
    \item There exists a path to goal states (liveness)
\end{enumerate}
\end{theorem>

\subsubsection{Bounded Model Checking}

\begin{theorem}[Bounded Safety Verification]
\label{thm:bounded_safety_verification}
For bounded time horizon $T$ and bounded state space $\mathcal{X}_B$:
$$\text{BMC}(\Sigma_{\text{HRI}}, \phi_{\text{safety}}, T) = \text{SAT} \Rightarrow \text{System is safe for } [0,T]$$

The computational complexity is:
$$\mathcal{O}(|\mathcal{X}_B|^T \cdot |\mathcal{A}|^T \cdot \text{complexity}(\phi_{\text{safety}}))$$
\end{theorem>

\subsection{Runtime Safety Monitoring}

\subsubsection{Online Safety Assessment}

\begin{definition}[Safety Monitor]
\label{def:safety_monitor}
The runtime safety monitor is a function:
$$\mathcal{M}: \mathcal{X} \times \mathcal{U} \times \mathbb{R}_+ \rightarrow [0,1]$$
that outputs the probability of safety over the next prediction horizon.
\end{definition>

\begin{theorem}[Runtime Safety Monitor Properties]
\label{thm:runtime_monitor_properties}
The safety monitor satisfies:
\begin{enumerate}
    \item \textbf{Conservatism}: $\mathcal{M}(x,u,t) \leq \Pr[\text{actual safety}]$
    \item \textbf{Responsiveness}: $\mathcal{M}$ can be computed in real-time (< 1ms)
    \item \textbf{Accuracy}: $|\mathcal{M}(x,u,t) - \Pr[\text{actual safety}]| \leq \epsilon_{\text{monitor}}$
\end{enumerate}
\end{theorem}

\subsubsection{Predictive Safety Assessment}

\begin{theorem}[Predictive Safety Horizon]
\label{thm:predictive_safety}
The system can predict safety over horizon $H$ with accuracy:
$$\Pr[\text{safe}_{[t,t+H]} = \mathcal{M}(x(t), u_{t:t+H}, t)] \geq 1 - \delta_H$$
where $\delta_H$ increases with prediction horizon due to uncertainty accumulation.
\end{theorem>

\subsection{Safe Learning and Adaptation}

\subsubsection{Safe RL with Barrier Constraints}

\begin{theorem}[Safe Reinforcement Learning]
\label{thm:safe_rl}
Modified RL objective with barrier constraints:
$$\max_\pi \E\left[\sum_{t=0}^{\infty} \gamma^t r(s_t, a_t)\right] \text{ s.t. } B(s_t, a_t) \geq 0 \quad \forall t$$

This ensures safety during exploration while maintaining convergence:
$$\text{Regret}_{\text{safe}}(T) \leq \text{Regret}_{\text{standard}}(T) + \mathcal{O}(\log T)$$
\end{theorem>

\subsubsection{Safe GP Learning}

\begin{theorem}[Safe Gaussian Process Updates]
\label{thm:safe_gp_updates}
GP updates preserve safety if:
\begin{enumerate}
    \item New observations are from the safe region: $(x_{\text{new}}, y_{\text{new}}) \in \mathcal{S} \times \mathcal{Y}_{\text{safe}}$
    \item Hyperparameter updates maintain calibration: $\Pr[\text{miscalibration}] \leq \delta_{\text{cal}}$
    \item Posterior samples respect safety constraints with high probability
\end{enumerate}
\end{theorem>

\subsection{Distributed and Multi-Robot Safety}

\subsubsection{Distributed Safety Verification}

\begin{theorem}[Distributed Safety Protocol]
\label{thm:distributed_safety}
For $N$ robots with communication graph $\mathcal{G}$, distributed safety is achieved if:
\begin{enumerate}
    \item Local safety: Each robot $i$ maintains $B_i(x_i, t) \geq 0$
    \item Communication safety: State information is shared within delay bound $\Delta_{\text{comm}}$
    \item Consensus safety: Robots agree on shared safety constraints
\end{enumerate}

The overall safety probability is:
$$\Pr[\text{system safe}] \geq \min_i \Pr[\text{robot } i \text{ safe}] - N \Delta_{\text{comm}} L_{\text{coupling}}$$
\end{theorem>

\subsubsection{Fault-Tolerant Safety}

\begin{theorem}[Safety Under Robot Failures]
\label{thm:fault_tolerant_safety}
The multi-robot system maintains safety under up to $f$ robot failures if:
$$N \geq 2f + 1 \text{ (Byzantine fault tolerance)}$$
and safety margins are increased:
$$\tilde{B}_i(x,t) = B_i(x,t) - \epsilon_{\text{fault}} \geq 0$$
\end{theorem}

\subsection{Real-Time Safety Guarantees}

\subsubsection{Worst-Case Execution Time Analysis}

\begin{theorem}[Real-Time Safety Verification WCET]
\label{thm:wcet_safety}
The worst-case execution time for safety verification is bounded:
\begin{align}
\text{WCET}_{\text{safety}} &\leq \text{WCET}_{\text{GP}} + \text{WCET}_{\text{MPC}} + \text{WCET}_{\text{monitor}} \\
&\leq C_1 n^3 + C_2 N^3 + C_3 \\
&= \mathcal{O}(\max(n^3, N^3))
\end{align}
where $n$ is GP training set size and $N$ is MPC horizon.
\end{theorem>

\subsubsection{Hard Real-Time Safety}

\begin{theorem}[Hard Real-Time Safety Guarantees]
\label{thm:hard_realtime_safety}
If the safety verification completes within deadline $D$:
$$\text{WCET}_{\text{safety}} \leq D - \epsilon_{\text{margin}}$$
then safety is guaranteed with probability 1, assuming deterministic worst-case bounds.
\end{theorem>

\subsection{Safety Validation and Testing}

\subsubsection{Statistical Validation}

\begin{theorem}[Statistical Safety Validation]
\label{thm:statistical_validation}
To validate safety probability $P_{\text{safe}} \geq 1 - \alpha$ with confidence $1 - \beta$:
$$N_{\text{tests}} \geq \frac{\log(\beta)}{\log(1 - \alpha)}$$
test scenarios are required with zero failures observed.
\end{theorem>

\subsubsection{Simulation-Based Verification}

\begin{theorem}[Monte Carlo Safety Verification]
\label{thm:monte_carlo_verification}
Using $N$ Monte Carlo simulations, the safety probability estimate has error:
$$|\hat{P}_{\text{safe}} - P_{\text{safe}}| \leq z_{\alpha/2} \sqrt{\frac{P_{\text{safe}}(1 - P_{\text{safe}})}{N}}$$
with confidence $1 - \alpha$.
\end{theorem>

\subsection{Integration Safety Assurance}

\subsubsection{Component Interaction Safety}

\begin{theorem}[Safe Component Integration]
\label{thm:safe_integration}
The integrated system is safe if:
\begin{enumerate}
    \item Each component is individually safe
    \item Component interfaces preserve safety invariants
    \item Information delays are bounded: $\tau_{\text{delay}} \leq \tau_{\text{max}}$
    \item Integration protocols handle failures gracefully
\end{enumerate}

The integrated safety probability satisfies:
$$P_{\text{integrated}} \geq \prod_i P_i - \epsilon_{\text{coupling}}$$
\end{theorem}

This comprehensive safety verification framework ensures that the integrated Model-Based RL system operates safely in human-robot interaction scenarios, providing mathematical guarantees at multiple levels from component-wise safety to system-wide safety assurance.